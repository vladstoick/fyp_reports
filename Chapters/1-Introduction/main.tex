\chapter{Introduction} \label{ch:introduction}
\section{Project motivations}
With the technological advancements of the 21st century, acquiring and interpreting
data has become essential for almost all areas of the economy. The \cite{ec:big_data} argues that ''good use of data is at
the centre of the future knowledge economy''.
Furthermore, the \cite{gov:big_data} also believes ``data can truly be a catalyst for a society, an economy, a country that works for everyone''. With this in mind, more and more companies acknowledge the importance of analyzing data and employing data analysts in their companies. It is estimated that the US needs over 140.000 workers with ``data analytical expertise'' \citep{Lohr2012}.

One of the most popular data sources in any company is usually represented by the main database, which stores various data. 6 out of 10 most popular databases are Relation DBMS \citep{db_engine:statistics}.
Accessing data in an RDBMS is done through \textit{Standard Query Language}
(SQL); therefore, course providers recognize the need to introduce programs that teach their students how to use SQL, in order to make them more employable.

Nowadays, SQL is being taught to almost all students in a Computer Science
degree. This project will explore how assessing student's SQL
assignments can be automated and become more accurate. There have been many attempts at improving this process made by various scholars, such as \cite{literature:activesql}, \cite{literature:assesql},
\cite{literature:sqlify}, \cite{literature:xdata}. This project builds on the existing work of these scholars, especially that of \cite{literature:xdata} in their project - \textbf{XData}.

\section{SQL assignments} \label{ch:introduction:assignments}

An important part of teaching any module is represented by assessments. They motivate the students to learn throughout the year and often promote deep learning, allowing students to get a better understanding of the material \citep{literature:assement}.

Another important role played by assessments throughout the year is the guidance offered to students based on the feedback provided on their earlier work \citep{literature:assement}. According to \cite{literature:assement}, the better the feedback provided, the greater the likelihood of students improving their performance. Effective feedback can help students understand what mistakes they make and also provides information about how to fix them \citep{literature:assement}.

With all these advantages in mind, it is obvious that SQL assignments are essential to success. SQL assessments usually consist of the following components:
\begin{itemize}
    \item A SQL table schema provided by the teacher. The schema includes
    the structure of the database tables and any relations between them;
    \item A SQL query that inserts the initial data in the database;
    \item A description of the assessment;
    \item A correct SQL query solution for the
    assignment.
\end{itemize}

The assessment requires a SQL query from the student that fits the requirements mentioned in the description of the problem. The query should return the same results as the query provided by the teacher. In general, these assignments are marked manually by Instructors or Teaching Assistants (TAs) \citep{literature:xdata}. Marking SQL manually is a very time consuming process which often leads to a very long time between the submission of work and students receiving feedback on the assignment. If the time between those two milestones is too long, it is often the case that students will not be able to acknowledge or understand their mistakes in time for future assignments. This represents a crucial problem when considering the importance of learning from mistakes or difficulties \citep{literature:assement}.

As previously mentioned, attempts at automating this process have been made. Various tools have been built, which are presented in more detail in the next chapter. However, most of these tools are only deployed in the courses taught by the authors, with no popular tool widely used in academia. Automating the assessment process brings multiple benefits:
\begin{itemize}
    \item It can give more time for TAs and teachers to focus on better helping students, rather than spending a great amount of time grading assignments.
    \item It opens the possibility of introducing more assignments throughout the course, which would further enhance students' learning process. Although it is generally better to provide less assignments which are more meaningful due to time constraints. \citep{literature:assement}, automating this process means that time constraints are no longer relevant.
    \item It can provide instant feedback on assignments. As mentioned earlier, feedback can help students improve throughout the course. In addition, instant feedback opens the possibility for a new type of assignment, one which is not part of the grade, but instead allows the student to practice their skills. This "practice" mode has been used by multiple commercial tools, allowing anyone to learn SQL by practicing and receiving instant feedback. The level of feedback is essential however, as a simple correct / incorrect might not be of very much help.
\end{itemize}


\section{Project aims} \label{ch:introduction:sec:project_aims}

The main aim of this project is to provide a tool that can automate grading of students' tasks in SQL. In particular, the tool will produce individual grades and feedback for students who have completed a SQL assignment. This tool will be able to provide hints or suggestions when students make errors (for instance, suggest if an user is missing a table in their query, or if they are selecting the wrong columns).

The project will also produce a web application where teachers will be able to create assignments for students, and where students will be able to provide solutions to them. The student will see instant feedback on their work with hints provided if they make mistakes. In addition, teachers will be able to see a breakdown of students' assignments based on the components of a query.

\section{Report structure}

This report will begin with an overview of the available commercial applications for grading SQL, as well as a literature review, looking at existing work done on automating assessing SQL in academia.

The next chapter will present the requirements of the applications, which will be separated in two parts: the requirements for the web application, and the requirements for the grading algorithm. The chapter will then continue with the specifications section, which will first discuss the selection of programming language and the reason for choosing it, then present the development process, and finally present individual specifications for each requirement.

The fourth chapter of the report will be the design one. The chapter will present the overall system architecture before moving on to the design of each component of the project (the web application and the library), and concluding with a section about the integration between the two components.

The following chapter will discuss the implementation and testing of the application. This chapter will cover the most important and relevant implementation details from both the web application and the library. The main focus will be on the grading algorithm as this is the core output of the project. The chapter will end with a discussion of the testing mechanisms used to ensure that software runs according to the requirements.

The sixth chapter will present how the project was evaluated. We will look at three important indicators: how the scope was attained, how the project compares with existing tools, and empirically evaluate the project using exercises from various data sources.

The next chapter will look at legal, social, ethical and professional issues associated with the project. The chapter will begin with a section that looks at the issues related to the use of automated grading and its implications. The next section will present how the project follows both BCS's Code of Conduct and BCS's Code of Good Practice, concluding with a section discussing the use of open-source libraries.

The 8th and last chapter of the main body of the report will present the conclusion of the project. The chapter will present an overview of the work achieved, its limitations and suggested future work.