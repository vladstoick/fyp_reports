\section{Commercial tools for learning SQL} \label{ch:lit:sec:tutor:comercial}

As mentioned before, SQL is one of the most in-demand skills
nowadays. This led to the appearence of multiple commercial 
applications that assess SQL. Most of them are used for helping people
learn SQL.

\subsubsection{HackerRank}
HackerRank is one of the most popular tools in the tech recruitment world,
which also allows SQL questions. The application fits most, but not all of our
needs. First of all, the application is closed-source therefore it cannot be extended. Second of all, the application does not allow regular
users to create new challenges. However, the most important aspect is
that it does not allow partial grading and only supports exact matching
of results. Moreover, no suggestions or hints are given in case of errors.

\subsubsection{LeetCode}
LeetCode is another popular tool in the recruitment world, very similar to
HackerRank. This means the application is still closed-source and
normal users cannot create the type of exercises we want.

\subsubsection{CodeCademy}
CodeCademy, one of the most popular tools for self-learning in Computer
Science, also provides a SQL course. CodeCademy provides an interactive
course for learning SQL and gives users multiple assignments.
Similarly to LeetCode and HackerRank, it is a closed source tool so it cannot be
extended. However, this specific tool provides partial grading and suggestions.

\subsubsection{Other online tools}
There are also other tools, albeit less popular, that share the same functionality with LeetCode, HackerRank and CodeCademy.
\begin{itemize}
    \item w3resource
    \item sqlzoo.net
    \item sqlbolt.com
\end{itemize}

\subsubsection{Conclusion}

\begin{center}
    \begin{tabularx}{\textwidth}{|*5{>{\centering\arraybackslash}X|}@{}}
        \hline
        \textbf{Name} & Open-Source & Users can create challenges & Partial grading & Provides hints \\
        \hline
        HackerRank & \xmark & \xmark & \xmark & \xmark \\
        \hline
        LeetCode & \xmark & \xmark & \xmark & \xmark \\
        \hline
        CodeCademy & \centering \xmark & \xmark & \cmark & \cmark \\
        \hline
        w3resource & \xmark & \xmark & \xmark & \xmark \\
        \hline
        sqlzoo.net & \xmark & \xmark & \xmark & \xmark \\
        \hline
        sqlbolt.com & \xmark & \xmark & \xmark & \xmark \\
        \hline
    \end{tabularx}
\end{center}

Unfortunately, while these tools represent great learning tools, due to their commercial and closed-source nature we are not going to be able to reuse anything from them. However, in the evaluation step of the project we will compare how well the tool compares to these commercial tools.