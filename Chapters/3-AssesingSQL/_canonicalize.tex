\section{Canonicalization of \texttt{SQL} queries}

As mentioned before, in order to compare the components of two \texttt{SQL} queries, we must first transform them to a common form (or standardize them) - a process called canonicalization \citep{literature:xdata}. This process will ensure that the queries will be as standardized as possible. It is worth mentioning, that while the process will standardize most queries, there are cases when it is simply impossible to canonicalize completely \textbf{EXAMPLES}

Each query goes through multiple transformations. We are going to look at each step in detail and explain how we transformed the query.

\subsection{Transforming \mintinline{mysql}{*} to the full list of columns}

One important transformation in comparing the two queries, is to first obtain the full list of columns used. To do this, we look at the columns of each element of the table expression, and see what columns they contain.

\subsection{Transforming equivalent columns}
