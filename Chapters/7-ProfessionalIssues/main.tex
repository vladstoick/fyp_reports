\chapter{Legal, Social, Ethical and Professional Issues}

\section{The use of automated grading}

As discussed in section \ref{ch:introduction:assignments}, manual grading in \texttt{SQL} is a very time-consuming process. It is no surprise, that there have been plenty of attempts at automating this process. While automatic grading of \texttt{SQL} is not widely used, automating grading of programming assignments is more common \citep{literature:assesment:automated:survey}. However, the most common marking technique is still the manual one \citep{literature:assesment:automated:survey}, even for programming assignments.

While in general all these automated tools bring many benefits which we also described in section \ref{ch:introduction:assignments}, there are some dangerous possible outcomes from the introduction of such tools. \cite{literature:assesment:automated:survey} found that the introduction of automation can lead to the following:
\begin{itemize}
    \item Teachers might incorporate these tools just because they believe they are relevant for the course, while also not increasing their workload. However, while more assignments are considered a good practice for students, the design of assignments is just as important if the student is to improve after receiving feedback. \citep{literature:assesment:automated:survey, literature:assement}.
    \item In a practice mode environment, students are more likely to submit work without checking or testing their work, only relying on the results of the application to do the work, knowing that they can resubmit for as many times as they want. \cite{literature:assesment:automated:survey} found that only 5\% of students designed their work before coding (for programming assignments). In addition, in such assignments some students do not feel motivated as it does not count for their grade \citep{literature:activesql}.
    \item Students can try to cheat the system if they find out what is graded. This is especially relevant for the \textit{tutor} applications we described in \ref{ch:lit:sec:tutor} where the grading is done based on the percentage of data matched.
\end{itemize}

In addition, \cite{literature:assesment:automated:brenda} noticed two additional problems while evaluating their \textit{Online Judge} system in a High School and a University course. They observed that some aspects (such as code style and maintainability of the code) can not be accurately graded by a machine. The second problem observed was that the level of feedback provided was not of the same quality compared to the one provided by a human.

Automated assessment also has the issue of potential false positive. After implementing a tool for grading \texttt{SQL}, \cite{literature:asqlg} surveyed the students about the use of such a tool in an exam. The majority of students said that such a tool should not be used (with over half saying it is not suitable at all) due to the potential existence of false positives.

Overall, the use of automated grading is still a sensitive subject. But as tools improve, their accuracy will also increase, and the likelihood of false positives will decrease.

\section{British Computing Society Code of Conduct}

The British Computing Society (BCS) publishes a \textit{Code of Conduct}\footnote{Available online on http://www.bcs.org/category/6030}. According to BCS this code of conduct has the goal of setting the required professional standards required to be a member of BCS. Throughout the development of the project we ensured that we are in compliance with the code of conduct. The code of conduct has four parts that the project built is fully compliant with

\begin{enumerate}
    \item \textbf{Public Interest}: the application built is non-discriminatory, does not affect public health, privacy, security and the environment. In addition, it promotes a equal access to the benefits of IT and the inclusion of all sectors of the society by providing an open-source tool that can allow anyone to develop their SQL skills. In addition, the right of 3rd parties is fully respected and their work in both this paper. In addition, the work done by others in the form of code is clearly indicated.
    \item \textbf{Professional Competence and Integrity}: all claims made in this paper are related to skills that were learned during the project. In addition, this project represented a chance to develop the professional knowledge on a continuous basis. No harm was done to others and no offers of bribery or unethical inducement were done.
    \item \textbf{Relevant Authority}: all work done in this project was done in accordance to the Relevant Authority (n.b. King's College London)
    \item \textbf{Duty to the Profession}: the project built will encourage and support fellow members in their professional development.
\end{enumerate}

\subsection{Code of Good Practice}

In addition to the code of conduct, BCS also publishes a \textit{Code of Good Practice}\footnote{Available online on http://www.bcs.org/upload/pdf/cop.pdf}. While this code is longer and provides more guidelines, most of them are relevant and are followed in the project. The most relevant section is especially Section 4.1 which relates to education. The relevant parts are the following:

\begin{itemize}
    \item \textit{Ensure students are equipped with the necessary underpinning to comprehend future developments.}: In a world that is becoming more-aware as discussed previously, SQL will be an essential skill in many jobs.
    \item \textit{Ensure that assessment is fair in its discriminatory function.}: the assessment method is non-discriminatory and is a standard one which is applied to all students.
    \item \textit{Ensure feedback to each student is sufficient to identify strengths and enable weaknesses to be addressed.}: the feedback provided should help students understand what type of mistakes they are making
    \item \textit{Develop yourself as a reflective and reflexive educational practitioner, building on student feedback as appropriate}: the feedback provided to teachers can help them better understand what knowledge their students lack or what aspects of SQL they are struggling with.
    \item \textit{Ensure students recognise the nature and unacceptability of plagiarism.}: while the project is not handling plagiarism, by having all data in a easily reportable format, future work could allow the application to automatically detect and report plagiarism.
\end{itemize}

\subsection{Open source library}

Both the library and the web application makes use of multiple open source libraries. As mentioned in the previous section, the work is clearly indicated in the Gemfile (the list of dependencies for a Ruby application) or in the gemspec (the lsit of dependencies for a Ruby library). The project only uses libraries that provide a license that is compatible with the work undertaken. In addition, both the library and the web application use a MIT license that allow anyone to reuse the application freely.
