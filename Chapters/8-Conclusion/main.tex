\chapter{Conclusion}

The aim of the project was to build a tool that can help automate the assessment of SQL assignments. For this, we have built a Ruby library and a Ruby on Rails web application that allows users to interact with the library. 

The final project represents a good foundation for future work on assessing SQL assignments.  However, as the evaluation chapter revealed, more work has to be done before it can be successfully used in an University course. Currently, with a success rate of about 70\%, the application is not close to being a reliable application that can be accurately used on its own. With more time spent, almost all issues discovered during the evaluation phase could be solved. Even after these issues are solved, one must overcome the legal, social, ethical and professional issues attached with the use of automated assessment that we discussed in chapter 7.

A tool such as the one we have built is perhaps better suited in an active learning environment where students can practice their SQL skills. Although XData has been used to grade assignments, we think that such algorithms can only work for basic queries. Once the queries become more advanced (e.g. multiple sub-queries, complicated joins, use of \mintinline{mysql}{CASE} statements, etc.), the existing algorithms (including the one built by this project) will have issues providing an accurate grade. However, it is worth mentioning that the algorithm used, which builds on top of XData, is likely the most accurate one of all existing tools with regards to its partial grading ability.

\section{Contributions}
The project has built a grading algorithm relatively similar to the one used in XData. Therefore, most contributions are related to improving XData's functionality.

\subsection{Improved partial grading}
The partial grading provided by this tool adds an extra layer of functionality on top of the partial grading provided by XData. While XData compares the individual parts of of the query's components, they are only looking for perfect matches (same name, same condition, same join type with same table etc.). On the other hand, the algorithm in this project uses a more advanced partial grading where small(er) mistakes (using the wrong join type, using a slightly different column, etc.) are not fully penalized. 

In addition, the grading of Boolean components by not only looking at the individual conditions, but also at the Boolean expression structure using the expression tree represents an important contribution.

\subsection{Ease of extending the application}
Another important contribution this project brings to the existing literature is related to the ease of extending the project. Out of the tools we looked at, only XData makes its source code public. Unfortunately, the code is clearly not meant to be open source. First of all, it doesn't provide any LICENSE which means that, legally, no one is able to reuse this code. Second of all, the code suffers from multiple problems: from files longer than 3000 lines to tens of lines of codes commented for no given reason, and a lack of proper documentation. Finally, the application lacks proper software testing, with only a few unit tests being included.

The tool built by the project adheres to the default standards used by \textit{RuboCop}. In addition, the library is composed of multiple small modules where each part handles one part of the grading process in order to ensure that the tool can be later extended. Furthermore, the grading library is fully documented using RDoc and Yard (instructions for generating the documentation are presented in the user manual). Finally, both the library and the web application have 100\% test coverage, made of unit tests and integration tests to ensure that new features added will not affect current functionality.

\section{Limitations}
\begin{itemize}
    \item No support for sub-queries: the application cannot handle sub-queries in its current version. Some initial work to implement this feature has been done, but it is yet to be finished;
    \item sql-parser brings multiple limitations. In our limited time we found and fixed some issues, but others remain:
    \begin{itemize}
        \item Poor and untested support for sub-queries;
        \item No support for aliases: there is no support for aliasing columns or tables;
        \item No support for other types of queries such as \mintinline{mysql}{DELETE}, \mintinline{mysql}{CREATE} etc.;
        \item No support for string functions;
        \item No support for math functions such as \mintinline{mysql}{ROUND}, \mintinline{mysql}{SQRT};
        \item It is likely that a more thorough testing will reveal even more limitations associated with the usage of sql-parser.
    \end{itemize}
    \item Hints are not very specific: they just point out what component is wrong, but not what is wrong with it.
    \item The evaluation performed has its limitations (presented in the evaluation chapter).
    \item The web application's design and user experience is fairly basic. Due to the higher importance of the grading algorithm, less time has been spent on the web application. Currently, the web application is simply an MVP (minimum viable product) that serves as a basic web interface for using the library.
\end{itemize}

\section{Future work}
\begin{itemize}
    \item \textbf{Moving away from sql-parser}: throughout the evaluation process, we have continuously observed how poorly implemented sql-parser is. We have noticed many issues such as missing features, the inability to parse SQL form other RDBMS, etc. Unfortunately, due to time constraints we had to stick with using sql-parser and try to fix it as much as possible. However, we observed how many issues are simply impossible to repair without a total rewrite. Therefore, any future work should start with moving away from sql-parser. One problem with moving away from sql-parser is that Ruby does not provide any alternative SQL parsers other than pg\_query which is also limited to a single RDBMS.
    \item \textbf{Implement a data generation system}: the work done by \citet{lit:xdata_d} in implementing a data generation system for XData showed that such an algorithm will perform better than fixed data sets. This will ensure that the chance of false positives (when the query is wrong but is marked as correct due to the results matching) will be almost zero.
    \item \textbf{Adding support for new types of queries}: while the \mintinline{mysql}{SELECT} query is one of the most important type of queries considering its use in data analysis, there are other types of queries that must be learned before one can use SQL on its own (e.g. \mintinline{mysql}{CREATE}, \mintinline{mysql}{DROP}, etc.). Currently, almost all applications that we looked at only support SELECT statements.
    \item \textbf{Adding support for relational algebra}: most University courses teach students relational algebra before SQL. By allowing student to practice relational algebra as well, they could further improve their knowledge about databases.
    \item \textbf{Adding support for new RDBMS}: while supporting MySQL allowed the project to focus on its core value (the grading algorithm), it also meant that no attention has been given to other database systems.
    \item \textbf{Make the web application more user-friendly}: many areas of the web application can be improved: user management, overall design, error display. The existing front-end has a very modern architecture that uses SASS, webpack, and Bootstrap.
    \item \textbf{Make the hints more relevant}: currently the hints can only look at what SQL component is wrong, but it cannot pinpoint exactly what is wrong within that component.
\end{itemize}