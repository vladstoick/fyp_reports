co\chapter{Legal, Social, Ethical and Professional Issues}

\section{The use of automated grading}

As discussed in section \ref{ch:introduction:assignments}, manual grading in \texttt{SQL} is a very time-consuming process. It is no surprise, that there have been plenty of attempts at automating this process. While automatic grading of \texttt{SQL} is not widely used, automating grading of programming assignments is more common \citep{literature:assesment:automated:survey}. However, the most common marking technique is still the manual one \citep{literature:assesment:automated:survey}, even for programming assignments.

While in general all these automated tools bring many benefits which we also described in section \ref{ch:introduction:assignments}, there are some dangerous possible outcomes from the introduction of such tools. \cite{literature:assesment:automated:survey} found that the introduction of automation can lead to the following:
\begin{itemize}
    \item Teachers might incorporate these tools just because they believe they are relevant for the course, while also not increasing their workload. However, while more assignments are considered a good practice for students, the design of assignments is just as important if the student is to improve after receiving feedback. \citep{literature:assesment:automated:survey, literature:assement}.
    \item In a practice mode environment, students are more likely to submit work without checking or testing their work, only relying on the results of the application to do the work, knowing that they can resubmit for as many times as they want. \cite{literature:assesment:automated:survey} found that only 5\% of students designed their work before coding (for programming assignments). In addition, in such assignments some students do not feel motivated as it doesn't count for their grade \citep{literature:activesql}.
    \item Students can try to cheat the system if they found out what is graded. This is especially relevant for the \textit{tutor} applications we described in \ref{ch:lit:sec:tutor} where the grading is done based on the percentage of data matched.
\end{itemize}

In addition, \cite{literature:assesment:automated:brenda} noticed two additional problems while evaluating their \textit{Online Judge} system in a High School and a University course. They observed that some aspects (such as code style, maintainability of the code) of the code can not be accurately graded by a machine. The second problem observed was that the level of feedback provided was not of the same quality compared to the one provided by a human.

Automated assessment also has the issue of potential false positive. After implementing a tool for grading \texttt{SQL}, \cite{literature:asqlg} surveyed the students about the use of such a tool in an exam. The majority of students said that such a tool should not be used (with over half saying it is not suitable at all) due to the potential existence of false positives.

Overall, the use of automated grading is still a sensitive subject. But as tools improve, their accuracy will also improve and the likelihood of false positives will decrease.

\section{British Computing Society Code of Conduct \& Code of Good Practice}

The British Computing Society publishes a \textit{Code of Conduct}\footnote{Available online on http://www.bcs.org/category/6030 } and a \textit{Code of Good Practice}\footnote{Available online on http://www.bcs.org/upload/pdf/cop.pdf}. This guides offer guidelines about how an IT professional should perform their job. Most guidelines are not related to the work undertaken in the project as they are designed for companies providing services to other. For the guidelines which are relevant (such as reasearch and security) we believe we are in full compliance.

\section{Open Source work}
The project uses multiple open source libraries. All libraries can be seen in the \texttt{Gemfile.lock} file from the root folder of both the library and the web application. All 3rd part libaries used provide a \texttt{LICENSE} that can be used freely for the purpose of this project and allows the project to be released to the public and modified by anyone.
