\chapter{Introduction}
\section{Project motivations}
With the technological advancements of the 21st century, acquiring and interpreting
data has become essential for almost all areas of the economy.
\cite{ec:big_data} ``argues that good use of data is at
the centre of the future knowledge economy''.
Furthermore, the \cite{gov:big_data} also believes that ``
Data can truly be a catalyst for a society,an economy,
a country that works for everyone.''. With this in mind, more and more companies are now employing data analysts in
their companies. For example, it is estimated that the US needs over 140.000 workers with
``data analytical expertise'' \citep{Lohr2012}.

One of the most popular data sources
in any company is usually the main database of the company which stores various
data. 6 out of 10 most popular databases are Relation \texttt{DBMS} \citep{db_engine:statistics}.
Accessing data in an \texttt{RDBMS} is done through \textit{Standard Query Language}
(\texttt{SQL}).

Nowadays, \texttt{SQL} is being taught to almost all students in a Computer Science
degree. In this project we will explore how we can make assessing student's \texttt{SQL}
assignments more accurate and computerized. There have been many attempts at improving this process made
by various scholars: \cite{literature:activesql}, \cite{literature:assesql},
\cite{literature:sqlify}, \cite{literature:xdata}. The work in this project is built
on top of the work presented by these scholars, especially the work presented by
\cite{literature:xdata} in their project \textbf{XData}.

\section{\texttt{SQL} assignments}
\texttt{SQL} assessments usually consist of the following components:

\begin{itemize}
    \item An \texttt{SQL} table schema provided by the teacher. The schema includes
    the structure of the database tables and any relations between them.
    \item Optionally, an \texttt{SQL} query that inserts the initial data in the database.
    This query is provided by the teacher.
    \item A description of the assessment and a correct query solution for the
    assignment.
\end{itemize}

The assessment requires an \texttt{SQL} query from the student that fits the requirements
mentioned in the description of the problem. The query should return the same
results as the query provided by the teacher.

There can be many types of query types (insertion, creation, updating, selecting
etc.), but in this project we will \textbf{only}
focus on \texttt{SELECT} statements.

\subsection{Problems with assessing \texttt{SQL} assignments}
Although comparing the results of two queries is a basic task, there are issues
in determining what is wrong if the results are different. Needless to say, just
looking at the results is not the best solution, considering the error might come
from a simple missing \texttt{WHERE} clause.

Therefore, when comparing the two queries we need to look at the structure
and elements of two queries: e.g. comparing what columns have been selected, or
what \texttt{WHERE} conditions have been used. This should generally return accurate
\textit{hints} to what is wrong.

However, in \texttt{SQL} there are often many ways to write the same query.
For instance, the following pairs will return the same results and they are
just as correct.

\begin{itemize}
  \item \mintinline{mysql}{SELECT A.b FROM t1 as A;}

  \mintinline{mysql}{SELECT b FROM table;}
  \item \mintinline{mysql}{SELECT t1.id FROM t1 JOIN t2 on t1.id = t2.id;}

  \mintinline{mysql}{SELECT t2.id FROM t1 JOIN t2 on t1.id = t2.id;}
  \item \mintinline{mysql}{SELECT a from t1 where a.id > 1;}

  \mintinline{mysql}{SELECT a from a.id >= 2;}
\end{itemize}

Therefore, we can see that a simple comparison of elements will not be efficient
in spotting if two queries are actually different. Therefore, before we can
compare the two queries, we need a way to transform the two queries to a
similar form. Without those transformations, the simple comparison may indicate
that two queries are very different.

\section{Project goals}

This project aims to improve automated assessment of \texttt{SELECT SQL} assignments
by providing a web application where teachers can create assignments and students can
submit solutions. The application will then provide a partial grade for each assignment,
obtained by comparing the query elements after applying a series of transformations in order
to have the queries in a standardized form.

Furthermore, the application resulted from this project is easily deployable across
various infrastructures and can be easily extended by adding new functionality.
