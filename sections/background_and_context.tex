\section{Databases and SQL}
A database is a collection of related data (data = known facts that can be
recorded and have an implicit meaning). A $RDBMS$ is a collection of programs
that allow users to create and maintain a database \cite{sql_course}.

$SQL$ (Structured Query Language) is a domain-specific language used in programming
and designed for managing data held in a relational database management system ($RDBMS$).
$SQL$ consists of many types of statements which can be classified, commonly, as: \cite{wiki:sql}
\begin{itemize}
    \item Data Query Language
    \item Data Definition Language
    \item Data Control Language
    \item Data Manipulation Language
\end{itemize}


\section{SQL Assesments}
$SQL$ assements usually consist of the following components:

\begin{itemize}
    \item A SQL table schema provided by the teacher. The schema includes
    the structure of the database tables and any relations between them.
    \item Optionally, an SQL query that inserts the initial data in the database.
    This query is provided by the teacher.
    \item A description of the assesment and a correct query solution for the
    assignment.
\end{itemize}

The assesment requires an $SQL$ query from the student that fits the requirements
mentioned in the description of the problem. The query should return the same
results as the query provided by the teacher.

\section{Project goals}
\textbf{\textit{[NC03]Automated Grading of SQL Tasks to Improve Student Learning}}
is a project whose main goal is to allow automatic grading of students' tasks
in $SQL$. In particular, this tool will produce individual grades and feedback
for students who have completed a $SQL$ assignment.

The main goal of the application is to grade students' $SQL$ assignments.
However, this tool will be able to provide hints or suggestions when students
make errors (for instance, suggest if an user is missing a table in their
query, or if they are selecting the wrong columns). This requires the
following functionality:
\begin{itemize}
    \item The ability to compare the results of two queries directly and check if the results are identical.
    \item The ability to compare internal elements of an $SQL$ query in case there are differences (e.g. tables selected, join conditions, $where$ filters, etc.)
    \item The ability to return "hints" or other useful information to users when there are differences between the students' results and the teacher's results.
\end{itemize}
