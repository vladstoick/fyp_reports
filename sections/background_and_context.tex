\section{Databases and SQL}
A database is a collection of related data (data = known facts that can be
recorded and have an implicit meaning). A relational database management system ($RDBMS$) is a collection of programs
that allow users to create and maintain a database \cite{sql_course}.

$SQL$ (Structured Query Language) is a domain-specific language used in programming
and designed for managing data held in an $RDBMS$.
$SQL$ consists of many types of statements which can be classified, commonly, as: \cite{wiki:sql}
\begin{itemize}
    \item Data Query Language
    \item Data Definition Language
    \item Data Control Language
    \item Data Manipulation Language
\end{itemize}


\section{SQL Assessments}
$SQL$ assessments usually consist of the following components:

\begin{itemize}
    \item An SQL table schema provided by the teacher. The schema includes
    the structure of the database tables and any relations between them.
    \item Optionally, an SQL query that inserts the initial data in the database.
    This query is provided by the teacher.
    \item A description of the assessment and a correct query solution for the
    assignment.
\end{itemize}

The assessment requires an $SQL$ query from the student that fits the requirements
mentioned in the description of the problem. The query should return the same
results as the query provided by the teacher.

\subsection{Comparing SQL queries}
Although comparing the results of two queries is a basic task, there are issues
in determining what is wrong if the results are different. Needless to say, just
looking at the results is not the best solution, considering the error might come
from a simple missing $WHERE$ clause.

Therefore, when comparing the two queries we need to look at the structure
and elements of two queries: e.g. comparing what columns have been selected, or
what $WHERE$ conditions have been used. This should generally return accurate
\textit{hints} to what is wrong. However, this solution will only work in a few
cases because there are multiple ways of writing the same query. For instance
the following two basic $SELECT$ queries are identical:
\begin{center}
    \texttt{SELECT a.b FROM table as A}

    \texttt{SELECT b FROM table}
\end{center}

That means that, before we compare the two tables, we need a way to transform
the two queries to a similar form. Without those transformations, the simple
comparison will return wrong results (for instance, in the query above
\texttt{a.b} in the first query and \texttt{b} in the second query, both refer
to the same column).


\section{Project goals}
\textbf{\textit{[NC03]Automated Grading of SQL Tasks to Improve Student Learning}}
is a project whose main goal is to allow automatic grading of students' tasks
in $SQL$. In particular, this tool will produce individual grades and feedback
for students who have completed a $SQL$ assignment. This tool will be able to
provide hints or suggestions when students make errors
(for instance, suggest if an user is missing a table in their query, or if they
are selecting the wrong columns).
