\newcommand{\cmark}{\ding{51}}%
\newcommand{\xmark}{\ding{55}}%
\section{Existing online tools for grading SQL}

\begin{center}
    \includegraphics[width=50mm,scale=0.5]{images/HackerRank.png}
\end{center}

HackerRank, one of the most popular tools in the tech recruiting world, also allows $SQL$ questions. The application fits most, but not all of our needs. First of all, the application is closed-source so we can not extend it at all. Second of all, the application does not allow regular users to create new challenges. However, the most important aspect is that it does not allow partial grading and it only supports exact matching of results. Moreover, no suggestions or hints are given in case of errors.

\begin{center}
    \includegraphics[width=50mm,scale=0.5]{images/LeetCode.png}
\end{center}

LeetCode, another popular tool in the recruiting world is very similar to HackerRank. That means that the application is still closed-source and normal users can't create the type of exercise we want.


\begin{center}
    \includegraphics[width=50mm,scale=0.5]{images/CodeCademy.png}
\end{center}

Codecademy, one of the most popular tools for self-learning in Computer Science, also provides an $SQL$ course. Codecademy provides an interactive course for learning $SQL$ and throughout it users have multiple assignments. Similarly to LeetCode and HackerRank it is a closed source tool so it can't be extended. However, this specific tool provides partial grading and suggestions.

\subsection*{Other less popular tools}
There are also other tools that share the same functionality with LeetCode and HackerRank
\begin{itemize}
    \item w3resource
    \item sqlzoo.net
    \item sqlbolt.com
\end{itemize}

\subsection*{Comparison of online tools}
We can now comapare the tools using the following criteria

\begin{itemize}
    \item C1 = Open-Source
    \item C2 = Allows user to create challenges
    \item C3 = Partial grading support
    \item C4 = Provides suggestions
    \item C5 = Provides existing database schema
\end{itemize}

\begin{center}
    \begin{tabular}{|c||c|c|c|c|c||}
        \hline
        \textbf{Name} & C1 & C2 & C3 & C4 & C5 \\
        \hline
        HackerRank & \xmark & \xmark & \xmark & \xmark & \cmark \\
        \hline
        LeetCode & \xmark & \xmark & \xmark & \xmark & \cmark \\
        \hline
        CodeCademy & \xmark & \xmark & \cmark & \cmark & \cmark \\
        \hline
        w3resource & \xmark & \xmark & \xmark & \xmark & \cmark \\
        \hline
        sqlzoo.net & \xmark & \xmark & \xmark & \xmark & \cmark \\
        \hline
        sqlbolt.com & \xmark & \xmark & \xmark & \xmark & \cmark \\
        \hline
    \end{tabular}
\end{center}


As no tool satisfies all our needs and because all tools are closed source, there is no way for us to use any existing only tool in any way in order to achieve the requirements for this project.

\section{Academic tools}
\subsection{XData}
Paper: \url{http://www.vldb.org/pvldb/vol9/p1541-chandra.pdf}

XData system is a \textit{automated and interactive plaform for grading student $SQL$ queries, as well for learning $SQL$} built by IIT Bombay. The tool is built in $Java$ and supports three types of databases: $PostgreSQl$, $Microsoft SQL$ and finally $Oracle DB$. The tool is made available at \url{https://gitlab.com/xdata/xdata-web}, however no $LICENSE$ for using the tool is provided, so that means we are not going to be able to extend or use this tool. XData repository on GitLab provides plenty of documentation about how to use this tool.

\subsubsection*{Partial grading}

XData fits most requirements of the project. Most importantly, it provides partial grading by comparing the student's query and the instructor's query. In order for this comparison to work, XData canonicalizes both queries in order to \textit{remove any irrelevant syntactic variations}.

\subsubsection*{Extending the tool}

Overall there are two big obstacles in using this tool for our purpose
\begin{enumerate}
    \item There is no $LICENCE$ which means that we are not allowed to extend or use this tool.
    \item The tool doesn't support $MySQL$ which is the database that's being teached at Kings College London.

However, there are many ideas about canonicalizing SQL queries that we can use in this project.
\end{enumerate}

\subsection{AsseSQL}

Paper: \url{http://crpit.com/confpapers/CRPITV20Prior.pdf}

AsseSQL is a similar tool to XData developed by Julia Coleman Prior from University of Techonology in Sidney.

\subsubsection*{Partial grading}
This approach doesn't provide the level of partial grading we want: it uses a simple pattern matching system for giving grades which can be unfair compared to the approach seen in XData.

\subsubsection*{Extending the tool}
No code is provided for this tool.

\subsection{ActiveSQL}

Paper: \url{http://webcache.googleusercontent.com/search?q=cache:PDEdmoa0YckJ:www.napier.ac.uk/~/media/worktribe/output-254045/pub20051pdf.pdf+&cd=5&hl=en&ct=clnk&gl=uk&client=safari}

ActiveSQL is an \textit{internet-based learning tool} for $SQL$ built by Dr. Gordon Rusell from Napier University, Edinburgh. ActiveSQL provides an interactive tutorial for learning $SQL$ that includes multiple exercies. Whenever submitting an answer to a question, a user receives a partial grade.

\subsubsection*{Partial grading}
The partial grading in ActiveSQL is built using multiple measures:
\begin{itemize}
    \item Accuracy rate: the proportion of correct answers. The answer is first compared to the instructor's query on a public data set, but its then also compared to a data set that's not made visible to the user.
    \item Hand-coded rules that check whether the student's query contains any weaknesses: e.g. a LIKE operand being used without any wildcards.
\end{itemize}

\subsubsection*{Extending the tool}
No code is provided for this tool.

\subsection{SQLify}
Paper: \url{https://pdfs.semanticscholar.org/eb57/08754c9e8eeec4ef96c5050d4a0e9d6141f7.pdf}

SQLify is another tool for assessing $SQL$ built at University of Southern Queensland.

\subsubsection*{Partial grading}
SQLify uses an approach based on peer reviews by students and reviews by students which is not the goal of this application.

\subsubsection*{Extending the tool}
No code is provided for this tool.
